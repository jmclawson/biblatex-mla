% !TEX TS-program = xelatex
\documentclass{article}
\usepackage[T1]{fontenc}
\usepackage[american]{babel}
\usepackage{csquotes}
% \setlength{\parindent}{0.5in}
\usepackage[style=mla]{biblatex}
\usepackage{hyperref}
\hypersetup{colorlinks,% 
citecolor=black,% 
% filecolor=black,% 
linkcolor=black,% 
urlcolor=black
}

\addbibresource{samples.bib}

\begin{document}
	
Here is a normal citation to an incollection work \autocite[7]{haggis99aa}. We will follow that up with a second citation to the same work \autocite[8]{haggis99aa}. Then we reference a book \autocite[194]{public08aa}. Finally, we reference an online work \autocite{Grammar-Girl2008}.

Here is a citation for a thesis \autocite[22]{webb84aa}. We will cite it a second time \autocite[23]{webb84aa}. Next is a citation for a film \autocite{jhabvala85aa}. We will follow that up with a reference for a reference entry \autocite{reference-noon89aa}. After that is a citation for an entire issue of a journal \autocite{appiah92aa}. Keeping up the theme of unusual entry types, we add a reference for an unpublished work \autocite{salviatiXXaa}. Next we cite a review \autocite[224]{slater01aa}. Finally we reference an anonymous review \autocite[785]{danish1972aa}.

Here is a citation for one work by an author with multiple works \autocite[12]{askme06aa}. We follow it up with a citation for a different work by the same author \autocite[34]{askme92aa}. Next, we show another call back to the first work \autocite[45]{askme06aa}, and we end it with a third reference to the first work \autocite[56]{askme06aa}. Here is a citation for a work with many authors \autocite[34]{Babich:2011dg}. And here is a second citation to the same work \autocite[32]{Babich:2011dg}.

Here's a citation for a native MLA-style, nonsemantic entry fashioned from containers \autocite{mla:shaw}. This ``containerized'' type is not generally recommended for widespread use, but it is included here to allow for maximum compatibility with the 9th edition of the \emph{MLA Handbook}. Here's a citation for a patent with the option enabled to show all authors in a citation and entry \autocite[12]{sorace}. Here's a citation for another patent without this option set \autocite[102]{laufenberg}. Here's a citation to a personal interview \autocite{misc:smith}, and here's one to an online lecture \autocite{elkm}.

A typical citation \autocite[12]{morrison02aa}. Immediately subsequent citation to the same source \autocite[34]{morrison02aa}. Immediately subsequent citations lacking page reference \autocite{morrison02aa}. Citation to a text by a prolific author \autocite[12]{frye57ab}. Subsequent immediate citations to the same source \autocite[34]{frye57ab}. Citation to new source, same author \autocite[56]{frye91aa}. Citation interrupting those by Frye \autocite[101]{morrison02aa}. Author tracker starts over \autocite[78]{frye91aa}. Suppressing Morrison's name for an entry with a single attribution to a given author prints only the page numbers \autocite*[102]{morrison02aa}. Suppressing the name of a prolific author like Frye will print enough information to avoid ambiguity \autocite*[91]{frye57ab}. Suppressing Morrison's name without a page number prints the title of the work \autocite*{morrison02aa}. Different author citation to reset trackers \autocite[91]{frye91aa}. Typical citation \autocite[12]{morrison02aa}. Citation using \verb|\mancite| to ignore the previous citation \mancite\autocite[34]{morrison02aa}.

\nocite{*}

\printbibliography
\end{document}
