% !TEX TS-program = xelatex
\documentclass{article}
\usepackage{fontspec}
\setmainfont{Times}
\usepackage[american]{babel}
\usepackage{csquotes}
% \setlength{\parindent}{0.5in}
\usepackage[style=mla]{biblatex}
\usepackage{hyperref}
\hypersetup{colorlinks,% 
citecolor=black,% 
% filecolor=black,% 
linkcolor=black,% 
urlcolor=black
}

\addbibresource{samples.bib}

\begin{document}

\section{Citations to Sources}
Here is a normal citation to an incollection work \autocite[7]{haggis99aa}. We will follow that up with a second citation to the same work \autocite[8]{haggis99aa}. Then we reference a book \autocite[194]{public08aa}. Finally, we reference an online work \autocite{Grammar-Girl2008}.

Here is a citation for a thesis \autocite[22]{webb84aa}. We will cite it a second time \autocite[23]{webb84aa}. Next is a citation for a film \autocite{jhabvala85aa}. We will follow that up with a reference for a reference entry \autocite{reference-noon89aa}. After that is a citation for an entire issue of a journal \autocite{appiah92aa}. Keeping up the theme of unusual entry types, we add a reference for an unpublished work \autocite{salviatiXXaa}. Next we cite a review \autocite[224]{slater01aa}. Finally we reference an anonymous review \autocite[785]{danish1972aa}.

Here is a citation for one work by an author with multiple works \autocite[12]{askme06aa}. We follow it up with a citation for a different work by the same author \autocite[34]{askme92aa}. Next, we show another call back to the first work \autocite[45]{askme06aa}, and we end it with a third reference to the first work \autocite[56]{askme06aa}. Here is a citation for a work with many authors \autocite[34]{Babich:2011dg}. And here is a second citation to the same work \autocite[32]{Babich:2011dg}.

Here's a citation for a native MLA-style, nonsemantic entry fashioned from containers \autocite{mla:shaw}. This ``containerized'' type is not generally recommended for widespread use, but it is included here to allow for maximum compatibility with the 9th edition of the \emph{MLA Handbook}. Here's a citation for a patent with the option enabled to show all authors in a citation and entry \autocite[12]{sorace}. Here's a citation for another patent without this option set \autocite[102]{laufenberg}. Here's a citation to a personal interview \autocite{misc:smith}, and here's one to an online lecture \autocite{elkm}.

Here's a book-length work by Tolstoy, showing a citation using non-Latin characters \autocite[22]{tolstoy:kreutzer}. Here's an entry with a general editor \autocite{crane69ab}; notice, too, that the publisher name is automatically truncated. Here's an essay-length work by Tolstoy with the title styled differently \autocite[22]{tolstoy:readers}. 

This is another typical citation \autocite[12]{morrison02aa}. This is an immediately subsequent citation to the same source \autocite[34]{morrison02aa}, followed by an example of immediately subsequent citations lacking page reference \autocite{morrison02aa}. This one shows a citation to a text by a prolific author \autocite[12]{frye57ab}. Subsequent immediate citations to the same source look like this \autocite[34]{frye57ab}. Citations to a different source, same author, look like this \autocite[56]{frye91aa}. A citation to a source by a different author will reset some of these trackers \autocite[101]{morrison02aa}. Another citation to a different author again resets the author tracker \autocite[78]{frye91aa}. Suppressing the author's name for someone like Morrison with only one source in the bibliography will print only the page numbers \autocite*[102]{morrison02aa}. Suppressing the name of a prolific author like Frye will print enough information to avoid ambiguity \autocite*[91]{frye57ab}. Suppressing Morrison's name without a page number prints the title of the work \autocite*{morrison02aa}. Different author citation to reset trackers \autocite[91]{frye91aa}. Typical citation \autocite[12]{morrison02aa}. Citation using \verb|\mancite| to ignore the previous citation \mancite\autocite[34]{morrison02aa}.

\section{Metadata from Sources}
In addition to the above citations, it is also possible to use commands within a sentence to reference specific metadata for sources without having to look up this metadata directly. Using correct styling, these commands will print the full or shortened names of a work's author(s), the full or shortened title, and the year of publication, respectively: \verb|\citeauthor{}|, \verb|\citeauthor*{}|, \verb|\citetitle{}|, \verb|\citetitle*{}|, and \verb|\citeyear{}|. The starred and unstarred variants of these commands might eventually swap places, so please keep this instability in mind with future updates.

This is the unstarred use of the first command to reference the full names of \citeauthor{appiah92aa}. It is followed by the starred use to cite just the last names of \citeauthor*{appiah92aa}. 

Here's an unstarred command to cite the title of a source with a defined \verb!shorttitle! field: \citetitle{wpa00aa}. Here's the starred version, which prints the long version of the title: \citetitle*{wpa00aa}. Here's the same effect using a work with no short title: \citetitle{librodehorasisabel}. And here's the starred variant of the same source, showing no change: \citetitle*{librodehorasisabel}. The command also works for a title which ought to be printed in quotation marks: \citetitle{Babich:2011dg}. Even though the command appears before the final period, this punctuation is pulled inside the quotation marks when using American localization.

Finally, here's an example showing the year alongside the title and author, using code to explain that our version of \citetitle{interview:amis} by \citeauthor{interview:amis} was published in \citeyear{interview:amis}.

\nocite{*}

\printbibliography
\end{document}
