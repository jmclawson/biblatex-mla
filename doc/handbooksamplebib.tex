% !TEX TS-program = xelatex
\documentclass{article}
% \usepackage[utf8]{inputenc}
% \usepackage[T1]{fontenc}
\usepackage{fontspec}
\setmainfont{Times}
\setsansfont{Helvetica}
% \setmonofont{Monaco}
\usepackage[style=mla,language=american]{biblatex}
\usepackage[hidelinks]{hyperref}
% \usepackage{libertine}

\addbibresource{handbooksamples.bib}

\usepackage{shortvrb}
\MakeShortVerb{\|}

% \usepackage{xcolor}
% \definecolor{work}{rgb}{0,0.7,0.7}
% \newcommand*{\checkbib}{\textbf{\textcolor{work}{|check bibliography:|}} }

\begin{document}

The list of Works Cited shows the printout of everything from the provided |.bib| file. The citations in section 1 allow for focused checking of specific types; click individual citations to jump directly to a particular bibliographic entry. Section 2 shows examples of some other commands in practice.

\section{Checking Types}
Here's an entry using the new |@mlasource| entrytype \autocite{mla:shaw}.

% Here's an entry for a published dissertation \autocite{dietze82aa}. Here's an article from an online database \autocite{social-media-family}. Here's an entry with an ISBN that is ignored in the list of Works Cited \autocite[150]{king}.

Here's an online lecture styled as a |@misc| entry \autocite{elkm}. Here's a similar lecture styled as an |@article| entry by the same author \autocite{elka}. And here's such a lecture as an |@unpublished| entry, which in part differs by printing the location in the list of Works Cited \autocite{elku}.

Here's an interview styled from a |@misc| entry \autocite{misc:smith}. Here's an untitled interview styled from an |@article| entry \autocite{interview:gaitskill}. Here's a titled interview with |type = {interview}| from a collection of interviews by one person \autocite{interview:amis}. 

Here's a comparison of these same three sources with slight changes beyond changes to titles. First is by changing the entrytype to article \autocite[34]{notmisc:smith}. Second is by dropping |type = {untitled}| \autocite{notinterview:gaitskill}. Third is by dropping |type = {interview}| \autocite{notinterview:amis}.

Here's an anonymous, unpublished entry with a defined |booktitle| field \autocite{librodehorasisabel}. Second is the same source without a |booktitle| field, but with |entrysubtype = {book}| defined instead \autocite{librodehorasisabel-2}.

Here's a citation whose entry in the Works Cited list shows its location \autocite{dewey99aa}.

Here's a book-length work by Tolstoy, showing a citation using non-Latin characters \autocite[22]{tolstoy:kreutzer}. Here's an essay-length work by him with the title styled differently \autocite[22]{tolstoy:readers}. 

Here's an entry with a general editor \autocite{crane69ab}. Notice, too, that the publisher name is truncated.

\section{Entry Metadata}

This is a starred use of |\citeauthor*| to reference the last names of \citeauthor*{appiah92aa}. It is followed by an unstarred use to cite the full names of \citeauthor{appiah92aa}. Here's |\citetitle| of a source with a defined a |shorttitle| field: \citetitle{wpa00aa}. Here's the starred version, which prints the long version: \citetitle*{wpa00aa}. Here's |\citetitle| of an essay-length source using non-Latin characters: \citetitle{tolstoy:readers}. Here's a work with no short title: \citetitle{librodehorasisabel}. And here's the starred variant of the same source, showing no change: \citetitle*{librodehorasisabel}. Finally, here's an example of |\citeyear| used beside the other commands, showing that our version of \citetitle{interview:amis} by \citeauthor{interview:amis} was published in \citeyear{interview:amis}.


\nocite{*}

\printbibliography
\end{document}
